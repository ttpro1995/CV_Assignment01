\documentclass[]{article}

%opening
\title{CS412 -- OpenCV Homework 01}
\author{Thai Thien - 1351040}

\begin{document}

\maketitle

\section{How to use}
\begin{itemize}
	\item i - show original image
	\item w - save file as img.png into current directory
	\item s - smooth image. Drag the top bar to change the amount
	\item S - A better way to smooth image. Drag the top bar to change the amount
	\item G or g - turn image into grayscale.
	\item c - display image in green, red, blue
	\item x - Sobel filter in x direction
	\item y - Sobel filter in y direction
	\item M or m - display magnitude of gradient. 
	\item r - rotate mode. Drag the track bar to rotate the image.
	\item q - quit
	\item h - display this message on console
\end{itemize}

\section{Display one channel of image}
The matrix of jpg image have shape (height, width, channel). The third dimension are channel, which is green, red, blue for [:,:,0], [:,:,1], [:,:,2]

To display one channel of a source image create empty matrix have same shape with image matrix. Get channel c from source[:,:,c] then put into [:,:,c] of our matrix.  

\section{Convert to grayscale}
Extract 3 channel from image. Then make new matrix shape(height, width) with each element is average of 3 channel from source image.

\section{Smooth}
Convolution source image with Gaussian kernel. Change sigma to change the amount of smooth.

\section{Derivative filter}
Convolution Sobel kernel with image. 

\section{Magnitude of the gradient}
Compute x, y derivative of the image using Sobel kernel. Then calculate magnitude of gradient using cv2.magnitude 


\section{Rotation}
Apply wrap affine to rotate image. Merge rotated image to old image so there is no hole. 

\end{document}
